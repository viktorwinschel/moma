
\documentclass{article}
\usepackage{amsmath, amssymb, tikz, graphicx}
\usepackage[margin=2.5cm]{geometry}
\usepackage{booktabs}

\title{Kategorientheoretisches Buchhaltungsmodell mit Makroinvarianz}
\author{Generated by ChatGPT for Viktor}
\date{2025-04-03}

\begin{document}
\maketitle

\section*{Theoretischer Teil}

Wir modellieren Agenten mit doppelter Buchführung als Objekte in einer Kategorie \( \mathcal{A} \).\\newline
Die Transaktionen zwischen Agenten werden als Morphismen in dieser Kategorie beschrieben. Die Kategorie ist wie folgt aufgebaut:

\begin{itemize}
  \item \textbf{Objekte}: MicroLedger \( L_i = (A_i, L_i) \) mit Aktiv- und Passivkonten
  \item \textbf{Morphismen}: MicroBookings \( b: A \to B \)
  \item \textbf{Pattern}: Diagramme aus diesen Morphismen
  \item \textbf{Bindings}: konkrete Werte wie Betrag und Kontenbindung
  \item \textbf{Colimit}: aggregierter Zustand, in dem alle Morphismen verklebt sind und die Makroinvarianz prüfen
  \item \textbf{Natürliche Transformation}: \( \mu: F \Rightarrow G \), mit \( \mu_i = \text{microledger\_balance} \)
\end{itemize}

\section*{Praxisbeispiel: Zwei Buchungen}

Wir betrachten zwei Transaktionen zwischen Händler und Bank:

\subsection*{1. Waren gegen Warenschein}

\begin{itemize}
  \item Agent A (Händler) liefert Ware (Aktiva)
  \item Agent B (Kunde) erhält Warenschein (Verbindlichkeit)
  \item Buchung: \( A: \, \text{debit 100} \rightarrow B: \, \text{credit 100} \)
  \item Pattern: \( A \xrightarrow{b_1} B \)
  \item Binding: Betrag = 100, Typ = goods/credit
  \item Colimit: \( \mu_A = +100, \mu_B = -100, \sum \mu_i = 0 \Rightarrow \text{Makroinvarianz erfüllt} \)
\end{itemize}

\subsection*{2. Geld gegen Kredit}

\begin{itemize}
  \item Agent A (Bank) vergibt Kredit in Geld
  \item Agent B (Kunde) erhält Geld, schuldet Kredit
  \item Buchung: \( A: \, \text{debit 50} \rightarrow B: \, \text{credit 50} \)
  \item Pattern: \( A \xrightarrow{b_2} B \)
  \item Binding: Betrag = 50, Typ = money/credit
  \item Colimit: \( \mu_A = +50, \mu_B = -50, \sum \mu_i = 0 \Rightarrow \text{Makroinvarianz erfüllt} \)
\end{itemize}

\section*{Rollenspezifikation}

\begin{description}
  \item[Buchhalter:] Jede Buchung ist ein Soll-/Haben-Vorgang in der doppelten Buchführung.
  \item[Programmierer:] Die Buchung ist eine Mutationsfunktion über Zustandsobjekte.
  \item[Kategorientheoretiker:] Morphismusdiagramme, Funktoren und Colimits prüfen strukturelle Konsistenz.
  \item[Investor:] Finanzflüsse sind sichtbar und quantitativ nachvollziehbar.
\end{description}

\end{document}
