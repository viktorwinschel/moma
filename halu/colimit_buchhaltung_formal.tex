
\documentclass{article}
\usepackage{amsmath, amssymb, tikz, graphicx}
\usepackage[margin=2.5cm]{geometry}
\usepackage{booktabs}
\usepackage{mathtools}
\usepackage{tikz}
\usepackage{tikz-cd}

\title{Kategorientheoretische Struktur einer doppelten Buchhaltung mit Colimits}
\author{für Viktor}
\date{\today}

\begin{document}
\maketitle

\section*{1. Formale Kategorientheorie der Buchhaltung}

Wir definieren eine Kategorie \( \mathcal{A} \), in der:

\begin{itemize}
  \item \textbf{Objekte} \( L_i \in \mathcal{A} \) sind \emph{MicroLedger}, d.h. Agenten mit interner Buchhaltung:
  \[
  L_i = (A_i, L_i) \quad \text{mit Assets und Liabilities}
  \]
  \item \textbf{Morphismen} \( b_{ij} : L_i \to L_j \) sind \emph{MicroBookings}, also Buchungen mit:
  \[
  b_{ij} = \text{Übertrag von Konto } A_i \text{ (debit) nach } L_j \text{ (credit)}
  \]
\end{itemize}

\paragraph{Pattern:} Ein Diagramm aus Morphismen,
\[
\begin{tikzcd}
L_i \arrow[r, "b_{ij}"] & L_j
\end{tikzcd}
\]

\paragraph{Binding:} Die konkrete Ausführung mit Betrag \( a \in \mathbb{R} \), z.B.
\[
b_{ij}(a): \quad \text{Sollbuchung } a \text{ bei } A_i, \quad \text{Habenbuchung } a \text{ bei } L_j
\]

\paragraph{Colimit:} Gegeben ein Diagramm \( D: \mathcal{J} \to \mathcal{A} \), der Colimit \( \operatorname{colim}(D) \) ist der konsistente aggregierte Ledger:
\[
\sum_i \mu_i = 0 \quad \text{mit } \mu_i = \text{microledger\_balance}(L_i)
\]

\paragraph{Natürliche Transformation:}
\[
\mu : F \Rightarrow G \quad \text{mit } \mu_i: L_i \to L_i \quad \text{und } \mu_i(x) = \text{Bilanz}(x)
\]

\section*{2. Beispiel: Zwei Buchungen}

\paragraph{Buchung 1:} Waren gegen Warenschein

\begin{itemize}
  \item \( L_1 \): Händler liefert Ware
  \item \( L_2 \): Kunde erhält Warenschein
\end{itemize}

\[
\begin{tikzcd}
L_1 \arrow[r, "b_1(100)"] & L_2
\end{tikzcd}
\qquad
\mu_1 = +100, \quad \mu_2 = -100, \quad \sum \mu_i = 0
\]

\paragraph{Buchung 2:} Geld gegen Kredit

\begin{itemize}
  \item \( L_3 \): Bank vergibt Geld
  \item \( L_4 \): Kunde erhält Kredit
\end{itemize}

\[
\begin{tikzcd}
L_3 \arrow[r, "b_2(50)"] & L_4
\end{tikzcd}
\qquad
\mu_3 = +50, \quad \mu_4 = -50, \quad \sum \mu_i = 0
\]

\section*{3. Interpretation}

\begin{description}
  \item[Buchhalter:] Jede \( b_{ij} \) ist ein Buchungssatz: Soll/Haben mit identischem Betrag.
  \item[Programmierer:] Jede \( b_{ij} \) ist eine Mutation von Zustandsobjekten mit Constraints.
  \item[Kategorientheoretiker:] Objekte + Morphismen + Colimit = konsistentes Finanzsystem.
  \item[Investor:] Jeder Fluss ist bilanziell transparent und nachvollziehbar.
\end{description}

\end{document}
